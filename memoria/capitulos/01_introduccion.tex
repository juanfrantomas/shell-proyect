\chapter{Introducción}

\section{Contexto}

En el contexto económico actual, el precio de los combustibles representa un factor significativo en el presupuesto de los hogares y empresas. La variabilidad de precios entre diferentes estaciones de servicio, incluso dentro de la misma provincia, puede suponer ahorros considerables para los consumidores que dispongan de información actualizada y fiable.

El Ministerio para la Transición Ecológica y el Reto Demográfico de España publica diariamente los precios de los carburantes de todas las estaciones de servicio del país a través de una API REST pública. Sin embargo, estos datos en formato bruto no son fácilmente accesibles ni interpretables para el usuario medio.

\section{Motivación}

Este proyecto nace de la necesidad de:

\begin{itemize}
  \item Automatizar la recopilación de datos de precios de combustibles
  \item Procesar y analizar grandes volúmenes de información
  \item Generar informes visuales y accesibles para los usuarios
  \item Identificar patrones y tendencias en los precios
\end{itemize}

\section{Estructura del documento}

Esta memoria se estructura en los siguientes capítulos:

\begin{itemize}
  \item \textbf{Capítulo 2}: Define los objetivos del proyecto
  \item \textbf{Capítulo 3}: Analiza el problema y las tecnologías disponibles
  \item \textbf{Capítulo 4}: Describe el diseño de la solución
  \item \textbf{Capítulo 5}: Detalla la implementación realizada
  \item \textbf{Capítulo 6}: Presenta los resultados obtenidos
  \item \textbf{Capítulo 7}: Expone las conclusiones y trabajo futuro
\end{itemize}
