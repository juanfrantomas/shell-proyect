\chapter{Análisis del problema}

\section{Fuente de datos}

\subsection{API del Ministerio}

El Ministerio para la Transición Ecológica proporciona una API REST pública con información actualizada de todas las estaciones de servicio de España.

\begin{itemize}
  \item \textbf{URL base}: \url{https://sedeaplicaciones.minetur.gob.es/ServiciosRESTCarburantes/}
  \item \textbf{Endpoint utilizado}: \code{PreciosCarburantes/EstacionesTerrestres/FiltroProvincia/46}
  \item \textbf{Formato}: JSON
  \item \textbf{Frecuencia de actualización}: Diaria
\end{itemize}

\subsection{Estructura de datos}

La respuesta JSON contiene:

\begin{lstlisting}[language=json,caption={Estructura de la respuesta JSON}]
{
  "Fecha": "11/10/2025 19:00:00",
  "ListaEESSPrecio": [
    {
      "IDEESS": "12345",
      "Rotulo": "NOMBRE ESTACION",
      "Direccion": "CALLE EJEMPLO 123",
      "Localidad": "VALENCIA",
      "Provincia": "VALENCIA",
      "Latitud": "39,4699",
      "Longitud (WGS84)": "-0,3763",
      "Precio Gasolina 95 E5": "1,549",
      "Precio Gasoleo A": "1,389"
    }
  ]
}
\end{lstlisting}

\section{Tecnologías evaluadas}

\subsection{Shell scripting (Bash)}

\textbf{Ventajas:}
\begin{itemize}
  \item Nativo en sistemas Linux
  \item Excelente integración con herramientas del sistema
  \item Bajo overhead y alta eficiencia
  \item Ideal para automatización con cron
\end{itemize}

\textbf{Desventajas:}
\begin{itemize}
  \item Sintaxis compleja para operaciones avanzadas
  \item Limitado en manipulación de estructuras de datos
\end{itemize}

\subsection{Python}

\textbf{Ventajas:}
\begin{itemize}
  \item Librerías robustas (requests, pandas, jinja2)
  \item Sintaxis clara y legible
  \item Mejor manejo de estructuras complejas
\end{itemize}

\textbf{Desventajas:}
\begin{itemize}
  \item Requiere instalación de dependencias
  \item Mayor consumo de recursos
\end{itemize}

\subsection{Decisión}

Se optó por \textbf{Bash + jq} por:
\begin{itemize}
  \item Mínimas dependencias (solo curl y jq)
  \item Integración nativa con el sistema
  \item Eficiencia en el procesamiento de texto
  \item Simplicidad de despliegue
\end{itemize}

\section{Requisitos funcionales}

\begin{enumerate}
  \item El sistema debe obtener datos actualizados de la API
  \item Debe validar la estructura del JSON recibido
  \item Debe calcular estadísticas de precios
  \item Debe generar rankings ordenados
  \item Debe producir informes en HTML y TXT
  \item Debe registrar todas las operaciones en logs
\end{enumerate}

\section{Requisitos no funcionales}

\begin{itemize}
  \item \textbf{Rendimiento}: Tiempo de ejecución < 30 segundos
  \item \textbf{Fiabilidad}: Control de errores en peticiones HTTP
  \item \textbf{Mantenibilidad}: Código documentado y modular
  \item \textbf{Portabilidad}: Compatible con distribuciones Linux estándar
\end{itemize}
