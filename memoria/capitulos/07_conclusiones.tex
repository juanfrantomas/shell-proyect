\chapter{Conclusiones y trabajo futuro}

\section{Conclusiones}

\subsection{Logros principales}

El proyecto ha conseguido desarrollar con éxito un sistema automatizado para el análisis de precios de combustibles que cumple todos los objetivos planteados:

\begin{enumerate}
  \item \textbf{Automatización completa}: El sistema funciona de forma autónoma mediante cron
  \item \textbf{Procesamiento eficiente}: Maneja grandes volúmenes de datos JSON con excelente rendimiento
  \item \textbf{Informes de calidad}: Genera reportes visuales y textuales informativos
  \item \textbf{Robustez}: Implementa validación de datos y manejo de errores
  \item \textbf{Mantenibilidad}: Código modular, documentado y versionado
\end{enumerate}

\subsection{Aprendizajes}

Durante el desarrollo del proyecto se han adquirido conocimientos en:

\begin{itemize}
  \item Programación avanzada en Bash
  \item Manipulación de datos JSON con jq
  \item Integración con APIs REST públicas
  \item Generación de contenido HTML dinámico
  \item Automatización de tareas con cron
  \item Buenas prácticas en shell scripting
\end{itemize}

\subsection{Ventajas de la solución}

\begin{itemize}
  \item \textbf{Ligereza}: Sin dependencias pesadas (Python, Node.js)
  \item \textbf{Portabilidad}: Compatible con cualquier sistema Linux/Unix
  \item \textbf{Eficiencia}: Bajo consumo de recursos
  \item \textbf{Integración}: Fácil incorporación en pipelines existentes
  \item \textbf{Transparencia}: Datos procesados guardados para auditoría
\end{itemize}

\subsection{Limitaciones identificadas}

\begin{itemize}
  \item No hay interfaz web interactiva
  \item Falta análisis histórico de tendencias
  \item Sin notificaciones a usuarios
  \item Limitado a una provincia
\end{itemize}

\section{Trabajo futuro}

\subsection{Mejoras a corto plazo}

\begin{enumerate}
  \item \textbf{Base de datos histórica}: Almacenar precios en SQLite para análisis temporal

  \item \textbf{Gráficas de tendencias}: Generar gráficos con gnuplot o matplotlib

  \item \textbf{Notificaciones}: Enviar alertas por email cuando los precios bajen

  \item \textbf{Múltiples provincias}: Extender el análisis a toda España

  \item \textbf{Más estadísticas}: Desviación estándar, percentiles, correlaciones geográficas
\end{enumerate}

\subsection{Mejoras a medio plazo}

\begin{enumerate}
  \item \textbf{API propia}: Exponer datos procesados mediante API REST

  \item \textbf{Interfaz web}: Dashboard interactivo con mapa de estaciones

  \item \textbf{App móvil}: Cliente móvil para consulta sobre la marcha

  \item \textbf{Geolocalización}: Encontrar estaciones más cercanas y baratas

  \item \textbf{Machine Learning}: Predecir tendencias de precios
\end{enumerate}

\subsection{Mejoras técnicas}

\begin{itemize}
  \item Implementar \code{set -euo pipefail} para mayor robustez
  \item Añadir lockfile para evitar ejecuciones concurrentes
  \item Escrituras atómicas con \code{mktemp}
  \item Retries con backoff exponencial en peticiones HTTP
  \item Tests automatizados con Bats
  \item Integración continua con GitHub Actions
  \item Linting con shellcheck
\end{itemize}

\section{Aplicabilidad}

El enfoque desarrollado es aplicable a otros dominios:

\begin{itemize}
  \item Análisis de precios de electricidad
  \item Monitorización de datos meteorológicos
  \item Seguimiento de indicadores económicos
  \item Agregación de datos de transporte público
\end{itemize}

\section{Reflexión final}

Este proyecto demuestra que es posible crear herramientas útiles y eficientes utilizando tecnologías simples y bien establecidas. La combinación de Bash, jq y herramientas Unix estándar ofrece una alternativa potente y minimalista a soluciones más complejas, siendo especialmente adecuada para entornos de servidor y automatización.

La experiencia adquirida en el desarrollo de este sistema es directamente transferible al ámbito profesional, donde la automatización, el procesamiento de datos y la generación de informes son tareas cotidianas.
