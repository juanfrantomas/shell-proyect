\chapter{REGISTRO Y AUDITORÍA}

Este apartado define cómo se audita cada ejecución y qué señales permiten juzgar su calidad. Leyendo el log puede saberse qué pasó, cuándo, en qué fase, con qué resultado y calidad de datos. Se usan mensajes con marca temporal (\texttt{[\%F \%T]}), taxonomía de fases alineada con el flujo, niveles simples (INFO, OK, ERROR) y línea de finalización inequívoca que refleja el código de salida. Se separan dos canales: \texttt{log.txt} (bitácora funcional del script) y \texttt{cron.log} (stdout/stderr de cron). Con esto, diagnosticar incidencias lleva segundos.

\textbf{Formato y fases:} Cada línea empieza por \texttt{[YYYY-MM-DD HH:MM:SS]}, sigue con fase entre paréntesis (Inicio, Descarga, Procesamiento, Indicadores, Informe, Finalizacion), nivel y mensaje claro. Las fases siguen el flujo, facilitando localizar problemas. Niveles: INFO (hitos), OK (chequeos superados), ERROR (fallos críticos con \texttt{exit 1}).

\begin{figure}[H]
  \footnotesize
  \begin{lstlisting}[language=bash]
# Ejemplos de mensajes en distintas fases
echo "[$(date '+%F %T')] (Inicio) INFO Identificador=$RUN_ID" 
     >> $NOMBRE_ARCHIVO_LOG

echo "[$(date '+%F %T')] (Descarga) OK Peticion Valida: 
     HTTP=$status" >> $NOMBRE_ARCHIVO_LOG

echo "[$(date '+%F %T')] (Procesamiento) OK JSON valido" 
     >> "$NOMBRE_ARCHIVO_LOG"

echo "[$(date '+%F %T')] (Indicadores) INFO Calculando Top 5" 
     >> "$NOMBRE_ARCHIVO_LOG"

echo "[$(date '+%F %T')] (Informe) OK Informe generado: $TXT" 
     >> "$NOMBRE_ARCHIVO_LOG"
\end{lstlisting}
  \caption{Formato de mensajes y fases del log}
\end{figure}

\textbf{Gestión de errores:} Cuando aparece ERROR, el criterio es una salida controlada: registrar el motivo e interrumpir con \texttt{exit 1}. Esto evita falsos positivos (informes con datos corruptos) y deja pista inequívoca. Ejemplos: HTTP no 2xx o JSON inválido.

\textbf{Finalización con trap:} Se usa \texttt{trap} para capturar código de salida y escribir línea de cierre. Si código es 0, indica éxito; si no, informa del código. Leer el último bloque de \texttt{log.txt} basta para saber si la ejecución terminó bien o mal.

\begin{figure}[H]
  \footnotesize
  \begin{lstlisting}[language=bash]
trap 'code=$?; if [ $code -eq 0 ]; then
  echo "[$(date "+%F %T")] (Finalizacion) INFO 
       Programa ejecutado correctamente" >> "$NOMBRE_ARCHIVO_LOG"
else
  echo "[$(date "+%F %T")] (Finalizacion) ERROR 
       Programa termino con codigo $code" >> "$NOMBRE_ARCHIVO_LOG"
fi' EXIT
\end{lstlisting}
  \caption{Finalización inequívoca con trap}
\end{figure}

\textbf{Separación de logs:} \texttt{log.txt} recoge lo que hace el script (fases, validaciones, indicadores); \texttt{cron.log} captura cómo se lanzó (errores de ruta, permisos, PATH incompleto). Esto diferencia problemas antes de entrar al script vs. durante ejecución.

\textbf{Señales de calidad:} El log anota el total de las estaciones, cuántas carecen de precio G95/Diésel y cuántas estan sin coordenadas. Si un día la media es extraña, el log permite atribuirlo a carencia de la fuente, no error en la lógica.

\textbf{Trazabilidad:} Todo se ancla en \texttt{RUN\_ID}. Aunque no aparece en cada línea, sí en mensajes clave (rutas de informes generados), permitiendo saltar del log al TXT/HTML y JSON crudo correspondiente para hacer la auditoría reproducible.

Con este esquema, el proyecto no solo descarga, valida, calcula e informa, también explica lo que hace de manera legible y verificable: resolución rápida de incidencias y confianza en que cada informe está respaldado por rastro claro desde el dato crudo hasta la línea final de éxito o error.
