\chapter{INFORME E INDICADORES}

Con el dataset validado y normalizado (Punto 4), este apartado se centra en medir y presentar. El objetivo es doble: calcular indicadores básicos pero informativos por combustible (cuántas estaciones tienen precio válido y cuáles son el mínimo, la media y el máximo) y ofrecer el resultado en dos formatos que cubren usos distintos: un TXT orientado a lectura rápida y un HTML más cómodo para consulta en pantalla. Todo mantiene la trazabilidad por \texttt{RUN\_ID} y deja hitos en el log, sin repetir la política de registro (Punto 6).

El cálculo parte de los subconjuntos \texttt{estaciones\_g95} y \texttt{estaciones\_diesel} preparados en el Punto 4. Para cada combustible se obtiene el conteo de estaciones con precio válido y, si el conteo es mayor que cero, se calculan mínimo, máximo y media aritmética. Esta precaución evita divisiones por cero y garantiza que, si el proveedor devuelve un bloque vacío, el informe lo refleje con un mensaje explícito y no con cifras espurias. El trabajo opera sobre valores numéricos reales (convertidos con \texttt{tonumber?} en el Punto 4), de modo que ordenar y agregar es directo y fiable.

\begin{figure}[H]
  \begin{lstlisting}[language=bash, caption={Figura 5.1 — Cálculo de métricas por combustible}]
# Metricas Gasolina 95
G95_COUNT=$(echo "$estaciones" | jq '
  [ .[] | .priceGasolina ] 
  | map(select(.!=null)) 
  | length')

if (( ${G95_COUNT:-0} > 0 )); then
  G95_MIN=$(echo "$estaciones" | jq '
    [ .[] | .priceGasolina ] 
    | map(select(.!=null)) 
    | min 
    | (. * 1000 | round / 1000)')
  
  G95_MAX=$(echo "$estaciones" | jq '
    [ .[] | .priceGasolina ] 
    | map(select(.!=null)) 
    | max 
    | (. * 1000 | round / 1000)')
  
  G95_AVG=$(echo "$estaciones" | jq '
    [ .[] | .priceGasolina ] 
    | map(select(.!=null)) 
    | if length>0 then 
        ((add/length) * 1000 | round / 1000) 
      else null end')
  
  echo "[...] (Informe) OK Estadisticos descriptivos obtenidos 
        para Gasolina 95"
else
  G95_MIN=""; G95_MAX=""; G95_AVG=""
  echo "[...] (Informe) ERROR No hay datos validos para 
        Gasolina 95"
fi
\end{lstlisting}
\end{figure}

Además de las estadísticas globales, el informe incluye una selección de estaciones con mejor precio. Se ordena por el campo de precio ascendente y se extrae un Top-5 (si hay menos de cinco, se muestran las disponibles). Al ser precios numéricos, la ordenación es estable y coherente. La intención no es construir un ránking complejo, sino dar una pista diaria de dónde mirar primero. Esta selección alimenta tanto el TXT como el HTML.

\begin{figure}[H]
  \begin{lstlisting}[language=bash, caption={Figura 5.2 — Selección del Top-5 por combustible}]
echo "[...] (Informe) INFO Calculando Top 5 Gasolina 95 
      por precio"

TOP5_G95=$(echo "$estaciones" | jq '
  [ .[] | select(.priceGasolina!=null)
    | { id, name, addr, lat, lon, priceGasolina }
  ]
  | sort_by(.priceGasolina)
  | .[0:5]
')

G95_TOP_N=$(echo "$TOP5_G95" | jq 'length' || echo 0)
if (( ${G95_TOP_N:-0} > 0 )); then
  echo "[...] (Informe) OK Ranking Top 5 Gasolina 95 generado 
        (registros=$G95_TOP_N)"
else
  echo "[...] (Informe) ERROR Sin datos validos para Top 5 
        Gasolina 95"
fi
\end{lstlisting}
\end{figure}

Para la presentación se unifica el formateo de precios. Internamente todo usa punto decimal; hacia el usuario se muestra con coma y tres decimales para mantener homogeneidad y no depender del locale. Se define una utilidad breve que transforma el número en cadena formateada; así, el mismo criterio se aplica en cabeceras y filas del Top-5, evitando discrepancias entre TXT y HTML.

\begin{figure}[H]
  \begin{lstlisting}[language=bash, caption={Figura 5.3 — Formateo de precios en jq}]
# Dentro del procesamiento con jq para redondear a 3 decimales:
(. * 1000 | round / 1000)

# Para sustituir punto por coma se puede usar gsub en la salida:
| tostring | gsub("\\."; ",")
\end{lstlisting}
\end{figure}

El informe TXT prioriza la lectura inmediata. Abre con una cabecera que muestra la fecha de ejecución (\texttt{FECHA\_CRON}, del Punto 3) y la fecha oficial del dataset (\texttt{FECHA\_API}, del Punto 4) y, a continuación, presenta dos bloques (Gasolina 95 y Diésel A) con el conteo de estaciones y las tres estadísticas (mín/med/max). Debajo, una lista numerada con el Top-5 indicando rótulo, municipio, precio y dirección. Si un combustible carece de precios válidos, se imprime un aviso claro (``No hay precios válidos…''). La estructura busca que, al abrir el fichero, sea evidente qué día y qué datos se ven y, en dos líneas, el rango de precios y por dónde empezar.

\begin{figure}[H]
  \begin{lstlisting}[language=bash, caption={Figura 5.4 — Generación del informe TXT (extracto)}]
generar_informe_txt() {
  if (( ${TOTAL_EESS:-0} > 0 )); then
    echo "[...] (Informe) INFO Generando informe en txt"
    {
      echo "=============================================="
      echo "   INFORME DE GASOLINERAS DE VALENCIA"
      echo "=============================================="
      echo ""
      echo "Fecha del cron:        $FECHA_CRON"
      echo "Fechas de la API:      $FECHA_API"
      echo ""
      echo "----------------------------------------------"
      echo "GASOLINA 95"
      echo "----------------------------------------------"
      echo "Precio minimo:         $G95_MIN EUR/L"
      echo "Precio maximo:         $G95_MAX EUR/L"
      echo "Precio medio:          $G95_AVG EUR/L"
      echo ""
      echo "Top 5 mas baratas:"
      echo "$TOP5_G95"
      # ... (sigue con Diesel)
    } > "$NOMBRE_ARCHIVO_INFORME_TXT"
    echo "[...] (Informe) OK Informe generado en: 
          $NOMBRE_ARCHIVO_INFORME_TXT"
  fi
}
\end{lstlisting}
\end{figure}

El informe HTML ofrece la misma información organizada en tablas y con una capa visual mínima. Incluye un encabezado con las dos fechas (\texttt{FECHA\_CRON} y \texttt{FECHA\_API}) y, para cada combustible, un título con insignias de min/med/max formateadas. Debajo, una tabla con el Top-5 (estación, municipio, precio con coma decimal y dirección). Cada fila recibe una clase según su relación con la media del combustible (\texttt{low}, \texttt{avg}, \texttt{high}) para resaltar lo más barato sin recurrir a gráficos ni dependencias externas. El HTML es autoinclusivo (estilos embebidos). Si un combustible no tiene datos válidos, se muestran cifras vacías y una tabla sin filas, en coherencia con el TXT.

\begin{figure}[H]
  \begin{lstlisting}[language=bash, caption={Figura 5.5 — Generación del informe HTML (extracto)}]
generar_informe_html() {
  echo "[...] (Informe) INFO Generando informe en html"
  
  cat > "$NOMBRE_ARCHIVO_INFORME_HTML" <<EOF
<!DOCTYPE html>
<html lang="es">
<head>
  <meta charset="UTF-8">
  <title>Informe de Gasolineras de Valencia</title>
  <style>
    body { font-family: sans-serif; margin: 20px; }
    .price.low { color: green; font-weight: bold; }
    .price.high { color: red; font-weight: bold; }
    .price.avg { color: black; }
    table { border-collapse: collapse; }
    th, td { border: 1px solid #ddd; padding: 8px; }
  </style>
</head>
<body>
  <h1>Informe de Gasolineras de Valencia</h1>
  <p>Fecha del cron: $FECHA_CRON</p>
  <p>Fechas de la API: $FECHA_API</p>
  
  <h2>GASOLINA 95</h2>
  <p>Minimo: <span class="price low">$G95_MIN EUR/L</span></p>
  <p>Maximo: <span class="price high">$G95_MAX EUR/L</span></p>
  <p>Medio: <span class="price avg">$G95_AVG EUR/L</span></p>
  
  <h3>Top 5 mas baratas:</h3>
  <table>
    <tr><th>ID</th><th>Nombre</th><th>Direccion</th>
        <th>Precio</th></tr>
$(echo "$TOP5_G95" | jq --arg avg "$G95_AVG" -r '
  .[] | "<tr><td>\(.id)</td><td>\(.name)</td>
         <td>\(.addr)</td>
         <td class=\"price " + 
         (if .priceGasolina < (\$avg | tonumber) then "low" 
          elif .priceGasolina > (\$avg | tonumber) then "high" 
          else "avg" end) + "\">
         \((.priceGasolina * 1000 | round / 1000))
         </td></tr>"')
  </table>
  <!-- ... (sigue con Diesel) -->
</body>
</html>
EOF
}
\end{lstlisting}
\end{figure}

Ambos informes se nombran con el \texttt{RUN\_ID} de la ejecución y se guardan bajo \texttt{informes/}. Así se preserva la trazabilidad de los Puntos 2 y 3: al ver un informe, se localizan el JSON crudo del mismo \texttt{RUN\_ID} y la traza correspondiente en el log. El script deja constancia explícita de la generación del TXT y del HTML con mensajes INFO, lo que permite comprobar de un vistazo —en el Punto 6— si la ejecución llegó a buen puerto y qué artefactos produjo.

\begin{figure}[H]
  \begin{lstlisting}[language=bash, caption={Figura 5.6 — Confirmación de informes generados}]
echo "[...] (Informe) OK Informe TXT generado en: 
      $NOMBRE_ARCHIVO_INFORME_TXT"

echo "[...] (Informe) OK Informe HTML generado en: 
      $NOMBRE_ARCHIVO_INFORME_HTML"
\end{lstlisting}
\end{figure}

Con esta capa de indicadores y presentación, el proyecto ofrece una salida diaria clara y utilizable sin repetir lógica de limpieza ni detalles de planificación. El Punto 6 cerrará el círculo describiendo el criterio de registro y calidad: qué se escribe en el log, cómo se distingue éxito de error y cómo localizar con rapidez una incidencia cuando algo no cuadra.
