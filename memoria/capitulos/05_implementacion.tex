\chapter{GENERACIÓN DE INFORMES}

Con el dataset validado y normalizado, este apartado mide y presenta. El objetivo: calcular indicadores básicos por combustible (conteo, mínimo, media, máximo) y ofrecer resultado en TXT para la lectura rápida y HTML para consultar en pantalla, manteniendo trazabilidad mediante el \texttt{RUN\_ID}.

\textbf{Cálculo de métricas:} Se parte de los subconjuntos preparados. Para cada combustible se obtiene conteo de estaciones con precio válido y, si es mayor que cero, se calculan mínimo, máximo y media aritmética. Esta precaución evita divisiones por cero. El trabajo opera sobre valores numéricos reales, haciendo ordenación y agregación directa y fiable.

Además de estadísticas globales, se incluye Top-5 con mejores precios. Se ordena por precio ascendente y se extraen las cinco mejores.

\begin{figure}[H]
  \footnotesize
  \begin{lstlisting}[language=bash]
# Metricas y Top-5
G95_COUNT=$(echo "$estaciones" | jq '
  [ .[] | .priceGasolina ] | map(select(.!=null)) | length')

if (( ${G95_COUNT:-0} > 0 )); then
  G95_MIN=$(echo "$estaciones" | jq '
    [ .[] | .priceGasolina ] | map(select(.!=null)) 
    | min | (. * 1000 | round / 1000)')
  G95_MAX=$(echo "$estaciones" | jq '
    [ .[] | .priceGasolina ] | map(select(.!=null)) 
    | max | (. * 1000 | round / 1000)')
  G95_AVG=$(echo "$estaciones" | jq '
    [ .[] | .priceGasolina ] | map(select(.!=null)) 
    | if length>0 then ((add/length) * 1000 | round / 1000) 
      else null end')
  
  TOP5_G95=$(echo "$estaciones" | jq '
    [ .[] | select(.priceGasolina!=null)
      | { id, name, addr, priceGasolina }]
    | sort_by(.priceGasolina) | .[0:5]')
fi
\end{lstlisting}
  \caption{Cálculo de métricas y Top-5 por combustible}
\end{figure}

\textbf{Formateo unificado:} Internamente se usa punto decimal; para el usuario se le muestra con coma y tres decimales para mantener homogeneidad. El mismo criterio se aplica en cabeceras y filas del Top-5, evitando discrepancias entre formatos:

\texttt{(. * 1000 | round / 1000) | tostring | gsub("\\textbackslash\\textbackslash."; ",")}

\textbf{Informe TXT:} Prioriza lectura inmediata. Inicia con cabeceras que muestran \\
\texttt{FECHA\_CRON} y \texttt{FECHA\_API}, seguidamente de dos bloques (Gasolina 95 y Diésel A) con conteo y estadísticas (mín/med/max). Debajo, muestra una lista numerada con las Top-5 indicando el nombre, precio y dirección.

\textbf{Informe HTML:} Ofrece misma información organizada en tablas con estilo. Incluye encabezado con fechas y, por combustible, título con insignias de min/med/max. La tabla muestra las columnas: ID, Nombre, Dirección, Latitud, Longitud, Precio Gasolina (€).

Ambos informes se nombran con \texttt{RUN\_ID} y guardan en \texttt{informes/}, preservando trazabilidad. El script deja constancia con mensajes INFO de la generación del TXT y HTML, permitiendo comprobar si la ejecución terminó bien y qué documenntos produjo.
