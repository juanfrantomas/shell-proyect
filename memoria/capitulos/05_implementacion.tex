\chapter{Implementación}

\section{Tecnologías utilizadas}

\begin{table}[H]
  \centering
  \begin{tabular}{@{}ll@{}}
    \toprule
    \textbf{Herramienta} & \textbf{Versión/Propósito} \\ \midrule
    Bash                 & 5.x - Shell scripting      \\
    curl                 & 7.x - Peticiones HTTP      \\
    jq                   & 1.6+ - Procesamiento JSON  \\
    cron                 & - Automatización           \\
    Git                  & Control de versiones       \\ \bottomrule
  \end{tabular}
  \caption{Stack tecnológico}
  \label{tab:stack}
\end{table}

\section{Detalles de implementación}

\subsection{Obtención de datos}

\begin{lstlisting}[style=bash,caption={Petición HTTP con curl}]
status=$(curl -sS -H "Accept: application/json" \
  -o "$NOMBRE_ARCHIVO_GUARDAR_ESTACIONES" \
  -w '%{http_code}' "$urlValencia" \
  2>> $NOMBRE_ARCHIVO_LOG)

if [ "$status" -ge 200 ] && [ "$status" -lt 300 ]; then
  echo "Peticion OK: Estado $status" | tee -a "$NOMBRE_ARCHIVO_LOG"
  getEstacionesValencia=$(cat "$NOMBRE_ARCHIVO_GUARDAR_ESTACIONES")
else
  echo "Error HTTP: $status" | tee -a "$NOMBRE_ARCHIVO_LOG"
fi
\end{lstlisting}

\subsection{Procesamiento con jq}

Transformación del JSON bruto a estructura simplificada:

\begin{lstlisting}[style=bash,caption={Transformación JSON}]
estaciones=$(echo "$getEstacionesValencia" | jq '[
  .ListaEESSPrecio[]
  | {
      id: (.IDEESS | tonumber?),
      name: .["Rotulo"],
      lat: ( (.Latitud // "") | gsub(",";".") 
           | (if . == "" then null else tonumber end) ),
      lon: ( (.["Longitud (WGS84)"] // "") | gsub(",";".") 
           | (if . == "" then null else tonumber end) ),
      addr: ( [ .["Direccion"], .["Localidad"], .["Provincia"] ] 
            | map(select(. != null and . != "")) 
            | join(", ") ),
      priceDiesel: ( (.["Precio Gasoleo A"] // "") | gsub(",";".") 
                   | (if . == "" then null else tonumber end) ),
      priceGasolina: ( ( .["Precio Gasolina 95 E5"]
                       // .["Precio Gasolina 95 E10"]
                       // .["Precio Gasolina 95 E5 Premium"]
                       // "" )
                       | gsub(",";".") 
                       | (if . == "" then null else tonumber end) )
    }
]')
\end{lstlisting}

\subsection{Cálculo de estadísticas}

\begin{lstlisting}[style=bash,caption={Estadísticas de precios}]
# Metricas gasolina con redondeo a 3 decimales
G95_MIN=$(echo "$estaciones" | jq '
  [ .[] | .priceGasolina ] 
  | map(select(.!=null)) 
  | min 
  | (. * 1000 | round / 1000)
')

G95_AVG=$(echo "$estaciones" | jq '
  [ .[] | .priceGasolina ] 
  | map(select(.!=null)) 
  | if length>0 then 
      ((add/length) * 1000 | round / 1000) 
    else null 
    end
')
\end{lstlisting}

\subsection{Generación de rankings}

\begin{lstlisting}[style=bash,caption={Top 5 más baratas}]
TOP5_G95=$(echo "$estaciones" | jq '
  [ .[] | select(.priceGasolina!=null) 
    | { id, name, addr, lat, lon, priceGasolina } 
  ] 
  | sort_by(.priceGasolina) 
  | .[0:5]
')
\end{lstlisting}

\section{Generación de informes HTML}

El informe HTML incluye:

\begin{itemize}
  \item CSS embebido con gradientes y sombras
  \item Tablas responsivas con hover effects
  \item Colores dinámicos basados en precio medio
  \item Estructura semántica (sections, headers, footer)
\end{itemize}

\subsection{Coloración dinámica de precios}

\begin{lstlisting}[style=bash,caption={Asignación de clases CSS}]
jq --arg avg "$G95_AVG" -r '
  .[] | "<tr><td>\(.id)</td><td>\(.name)</td>
         <td>\(.addr)</td><td>\(.lat)</td><td>\(.lon)</td>
         <td class='"'"'price " + 
         (if .priceGasolina < ($avg | tonumber) then "low" 
          elif .priceGasolina > ($avg | tonumber) then "high" 
          else "avg" end) + "'"'"'>
         \((.priceGasolina * 1000 | round / 1000))
         </td></tr>"
'
\end{lstlisting}

\section{Automatización con cron}

Ejemplo de entrada crontab para ejecución diaria:

\begin{lstlisting}[style=bash]
# Ejecutar a las 8:00 AM todos los dias
0 8 * * * /home/usuario/shell-proyect/analisis_json.sh
\end{lstlisting}

\section{Control de versiones}

Se utiliza Git para el control de versiones con la siguiente estructura de ramas:

\begin{itemize}
  \item \code{main}: Rama principal estable
  \item \code{feat/informe-html}: Desarrollo de informes HTML
  \item \code{feat/*}: Nuevas funcionalidades
\end{itemize}
