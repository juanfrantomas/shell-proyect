\chapter{FUENTE DE DATOS}

El sistema se apoya en datos públicos de precios de estaciones de servicio para Valencia, servidos en JSON mediante endpoint estable y accesible sin autenticación. Esto permite: (1) ejecución desatendida sin dependencia de formularios ni tokens, integración directa en cron; (2) formato JSON facilita verificación estructural y extracción precisa de campos antes de cualquier tratamiento.

\begin{figure}[H]
  \centering
  \footnotesize
  \begin{tabular}{ll}
    \toprule
    \textbf{Elemento} & \textbf{Detalle}                          \\
    \midrule
    Formato           & JSON                                      \\
    Método            & HTTP GET con curl                         \\
    Autenticación     & No requiere                               \\
    Frecuencia        & Actualización diaria                      \\
    Campos clave      & Precio G95/Diésel, coordenadas, dirección \\
    \bottomrule
  \end{tabular}
  \caption{Características de la fuente de datos}
\end{figure}

La obtención se realiza con petición HTTP idempotente mediante \texttt{curl} en modo silencioso, adecuada para cron. Se fija cabecera \texttt{Accept: application/json}, se vuelca respuesta a fichero y se captura código HTTP en variable \texttt{status} con \texttt{-w '\%\{http\_code\}'}. Con esa señal decidimos si proseguir o abortar: solo continuamos ante 2xx. Filosofía: ``fallar pronto y con diagnóstico comprensible''.

\begin{figure}[H]
  \footnotesize
  \begin{lstlisting}[language=bash]
# Descarga con curl y captura de codigo HTTP
status=$(curl -s -w '%{http_code}' -o "$ARCHIVO_JSON" \
  -H "Accept: application/json" "$URL_API")

if [ "$status" -ge 200 ] && [ "$status" -lt 300 ]; then
  echo "[...] (Descarga) OK HTTP=$status"
else
  echo "[...] (Descarga) ERROR HTTP=$status"
  exit 1
fi
\end{lstlisting}
  \caption{Descarga y validación HTTP}
\end{figure}

El fichero se guarda con identificador temporal (\texttt{RUN\_ID}) en el nombre, asegurando trazabilidad de extremo a extremo. Una ejecución concreta deja, bajo ese \texttt{RUN\_ID}, el JSON original, informes en TXT y HTML y entradas en el log. Si surge duda o anomalía, es posible reconstruir qué datos se usaron y qué cálculo se hizo.

La fuente incluye marca temporal propia (\texttt{Fecha}) del momento de actualización oficial. El sistema maneja dos referencias complementarias: fecha/hora de ejecución (cuando corre cron) y fecha API (cuándo se actualizaron datos en proveedor). En informes se muestran ambas para claridad.

Aspecto práctico: campos numéricos (precios) vienen con coma decimal. No afecta descarga, pero sí uso. Si no se normaliza a punto decimal y no se tipa a número, cálculos como medias, mínimos u ordenaciones fallan. Este punto se resuelve en Punto 4 con normalización y tipado mediante \texttt{jq}. La planificación (cadencia y redirecciones) se detalla en Punto 3.

Con fuente y mecánica definidas, falta encajar su lugar en el proyecto y repetición diaria. Para que cada descarga termine en directorio correcto, conserve \texttt{RUN\_ID} y deje rastro auditable, la ejecución no puede depender de acciones manuales ni variables de entorno cambiantes. El siguiente apartado define arquitectura mínima (árbol de carpetas y nombres) y automatización que la hace operativa.
