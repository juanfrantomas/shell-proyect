\chapter{Diseño de la solución}

\section{Arquitectura del sistema}

El sistema sigue una arquitectura de flujo de datos en pipeline:

\begin{enumerate}
  \item \textbf{Obtención}: Petición HTTP a la API
  \item \textbf{Almacenamiento}: Guardado del JSON bruto
  \item \textbf{Validación}: Comprobación de estructura y contenido
  \item \textbf{Transformación}: Procesamiento con jq
  \item \textbf{Análisis}: Cálculo de estadísticas
  \item \textbf{Generación}: Creación de informes
  \item \textbf{Registro}: Logging de operaciones
\end{enumerate}

\section{Componentes principales}

\subsection{Script principal: \code{analisis\_json.sh}}

Orquesta todo el proceso y contiene la lógica principal.

\textbf{Variables de configuración:}
\begin{itemize}
  \item \code{BASE\_DIR}: Directorio raíz del proyecto
  \item \code{urlValencia}: URL de la API
  \item \code{RUN\_ID}: Identificador único de ejecución
\end{itemize}

\subsection{Estructura de directorios}

\begin{lstlisting}[style=bash]
shell-proyect/
├── analisis_json.sh
├── datos/
│   └── estacionesValencia_YYYYMMDD-HHMMSS.json
├── informes/
│   ├── informe_YYYYMMDD-HHMMSS.html
│   └── informe_YYYYMMDD-HHMMSS.txt
├── planificacion/
└── log.txt
\end{lstlisting}

\section{Funciones principales}

\subsection{\code{crear\_carpetas()}}

Crea la estructura de directorios necesaria si no existe.

\subsection{\code{generar\_informe\_html()}}

Genera un informe visual en HTML con:
\begin{itemize}
  \item Estadísticas de precios
  \item Tablas con Top 5 estaciones más baratas
  \item Código de colores (verde: barato, rojo: caro)
  \item Diseño responsivo
\end{itemize}

\subsection{\code{generar\_informe\_txt()}}

Genera un informe en texto plano para consulta rápida en terminal.

\section{Flujo de datos}

\begin{figure}[H]
  \centering
  \begin{verbatim}
  API ─> curl ─> JSON bruto ─> jq ─> JSON procesado
                                         │
                                         ├─> Estadísticas
                                         ├─> Rankings
                                         └─> Informes (HTML/TXT)
  \end{verbatim}
  \caption{Flujo de procesamiento de datos}
  \label{fig:flujo}
\end{figure}

\section{Manejo de errores}

\subsection{Validación de entrada}
\begin{itemize}
  \item Comprobación de código HTTP (200-299)
  \item Validación de JSON con \code{jq empty}
  \item Verificación de campos obligatorios
\end{itemize}

\subsection{Registro de errores}
Todos los errores se registran en \code{log.txt} con timestamp y contexto.

\section{Optimizaciones}

\subsection{Caché local}
Se guarda cada respuesta JSON para permitir análisis offline sin nuevas peticiones.

\subsection{Rutas absolutas}
Uso de \code{BASE\_DIR} para independencia del directorio de trabajo actual.
