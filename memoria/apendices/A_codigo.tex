\chapter{Código fuente}

\section{Script principal}

El código completo del script \code{analisis\_json.sh} se encuentra en el repositorio del proyecto:

\url{https://github.com/juanfrantomas/shell-proyect}

\section{Fragmentos destacados}

\subsection{Función de validación JSON}

\begin{lstlisting}[style=bash,caption={Validación de estructura JSON}]
echo "Comprobando que la variable no este vacia" \
  >> $NOMBRE_ARCHIVO_LOG
if [ -z "$getEstacionesValencia" ]; then
  echo "getEstacionesValencia vacio. Termino." \
    >> $NOMBRE_ARCHIVO_LOG
  exit 1
fi

echo "Comprobando que el JSON es valido" \
  >> $NOMBRE_ARCHIVO_LOG
echo "$getEstacionesValencia" | jq empty > /dev/null \
  2>>$NOMBRE_ARCHIVO_LOG \
  || { echo "JSON invalido" >> $NOMBRE_ARCHIVO_LOG; exit 1; }

echo "Comprobando que existe la lista de estaciones" \
  >> $NOMBRE_ARCHIVO_LOG
echo "$getEstacionesValencia" \
  | jq -e '.ListaEESSPrecio | type=="array"' \
  >/dev/null 2>>$NOMBRE_ARCHIVO_LOG \
  || { echo "Falta .ListaEESSPrecio[]" \
       >> $NOMBRE_ARCHIVO_LOG; exit 1; }
\end{lstlisting}

\subsection{Generación de tabla HTML}

\begin{lstlisting}[style=bash,caption={Tabla HTML dinámica con colores}]
<table>
  <tr>
    <th>ID</th><th>Nombre</th><th>Direccion</th>
    <th>Latitud</th><th>Longitud</th>
    <th>Precio Gasolina (EUR)</th>
  </tr>
$(echo "$TOP5_G95" | jq --arg avg "$G95_AVG" -r '
  .[] | "<tr><td>\(.id)</td><td>\(.name)</td>
         <td>\(.addr)</td><td>\(.lat)</td><td>\(.lon)</td>
         <td class='"'"'price " + 
           (if .priceGasolina < ($avg | tonumber) 
            then "low" 
            elif .priceGasolina > ($avg | tonumber) 
            then "high" 
            else "avg" end) + "'"'"'>
         \((.priceGasolina * 1000 | round / 1000))
         </td></tr>"
')
</table>
\end{lstlisting}

\subsection{Configuración BASE\_DIR}

\begin{lstlisting}[style=bash,caption={Rutas absolutas con BASE\_DIR}]
#!/usr/bin/env bash

BASE_DIR="$(cd "$(dirname "${BASH_SOURCE[0]}")" && pwd)"

# Funciones necesarias
crear_carpetas() {
  for dir in datos informes planificacion; do
    path="$BASE_DIR/$dir"
    if [ ! -d "$path" ]; then
      mkdir -p "$path"
      echo "Carpeta creada: $path" | tee -a "$BASE_DIR/log.txt"
    else
      echo "Carpeta ya existe: $path" | tee -a "$BASE_DIR/log.txt"
    fi
  done
}
\end{lstlisting}

\section{Estructura del proyecto}

\begin{lstlisting}[style=bash]
shell-proyect/
├── analisis_json.sh          # Script principal
├── README.md                  # Documentacion
├── log.txt                    # Log principal
├── datos/                     # JSON descargados
│   ├── estacionesValencia_20251011-190935.json
│   └── ...
├── informes/                  # Informes generados
│   ├── informe_20251011-190935.html
│   ├── informe_20251011-190935.txt
│   └── ...
├── planificacion/             # Documentos de planificacion
└── memoria/                   # Esta memoria
    ├── main.tex
    ├── capitulos/
    ├── config/
    ├── imagenes/
    └── apendices/
\end{lstlisting}

\section{Comandos útiles}

\subsection{Ejecución manual}

\begin{lstlisting}[style=bash]
# Dar permisos de ejecucion
chmod +x analisis_json.sh

# Ejecutar el script
./analisis_json.sh

# Ver el log
tail -f log.txt

# Abrir ultimo informe HTML generado
xdg-open informes/informe_$(ls -t informes/ | head -1)
\end{lstlisting}

\subsection{Configuración de cron}

\begin{lstlisting}[style=bash]
# Editar crontab
crontab -e

# Añadir linea (ejecutar diariamente a las 8:00 AM)
0 8 * * * /ruta/completa/analisis_json.sh

# Ver tareas programadas
crontab -l
\end{lstlisting}

\subsection{Análisis de logs}

\begin{lstlisting}[style=bash]
# Ver errores
grep -i "error" log.txt

# Contar ejecuciones
grep -c "Empieza el programa" log.txt

# Ver ultimas estadisticas
grep "stats" log.txt | tail -5
\end{lstlisting}
