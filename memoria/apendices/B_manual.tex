\chapter{Manual de usuario}

\section{Requisitos del sistema}

\subsection{Software necesario}

\begin{itemize}
  \item Sistema operativo: Linux, macOS o WSL (Windows Subsystem for Linux)
  \item Bash: versión 4.0 o superior
  \item curl: para peticiones HTTP
  \item jq: para procesamiento JSON
  \item Git (opcional): para control de versiones
\end{itemize}

\subsection{Instalación de dependencias}

\textbf{En Ubuntu/Debian:}
\begin{lstlisting}[style=bash]
sudo apt-get update
sudo apt-get install curl jq
\end{lstlisting}

\textbf{En Fedora/RHEL:}
\begin{lstlisting}[style=bash]
sudo dnf install curl jq
\end{lstlisting}

\textbf{En macOS:}
\begin{lstlisting}[style=bash]
brew install curl jq
\end{lstlisting}

\section{Instalación del proyecto}

\subsection{Clonar el repositorio}

\begin{lstlisting}[style=bash]
git clone https://github.com/juanfrantomas/shell-proyect.git
cd shell-proyect
\end{lstlisting}

\subsection{Dar permisos de ejecución}

\begin{lstlisting}[style=bash]
chmod +x analisis_json.sh
\end{lstlisting}

\section{Uso básico}

\subsection{Ejecución manual}

Para ejecutar el script una sola vez:

\begin{lstlisting}[style=bash]
./analisis_json.sh
\end{lstlisting}

El script:
\begin{enumerate}
  \item Creará las carpetas necesarias (datos, informes, planificacion)
  \item Descargará los datos de la API
  \item Procesará la información
  \item Generará los informes en \code{informes/}
  \item Registrará las operaciones en \code{log.txt}
\end{enumerate}

\subsection{Visualizar resultados}

\textbf{Abrir el informe HTML:}
\begin{lstlisting}[style=bash]
# En Linux
xdg-open informes/informe_$(ls -t informes/*.html | head -1 | xargs basename)

# En macOS
open informes/informe_$(ls -t informes/*.html | head -1 | xargs basename)
\end{lstlisting}

\textbf{Ver el informe TXT:}
\begin{lstlisting}[style=bash]
cat informes/informe_$(ls -t informes/*.txt | head -1 | xargs basename)
\end{lstlisting}

\section{Automatización}

\subsection{Configurar ejecución diaria}

\textbf{Paso 1:} Editar el crontab
\begin{lstlisting}[style=bash]
crontab -e
\end{lstlisting}

\textbf{Paso 2:} Añadir la siguiente línea (ajustar la ruta):
\begin{lstlisting}[style=bash]
# Ejecutar todos los dias a las 8:00 AM
0 8 * * * /home/usuario/shell-proyect/analisis_json.sh

# Ejecutar cada 6 horas
0 */6 * * * /home/usuario/shell-proyect/analisis_json.sh

# Ejecutar de lunes a viernes a las 9:00 AM
0 9 * * 1-5 /home/usuario/shell-proyect/analisis_json.sh
\end{lstlisting}

\textbf{Paso 3:} Verificar que se ha añadido correctamente
\begin{lstlisting}[style=bash]
crontab -l
\end{lstlisting}

\section{Interpretación de resultados}

\subsection{Informe HTML}

El informe HTML contiene:

\begin{itemize}
  \item \textbf{Fecha de ejecución}: Momento en que se generó el informe
  \item \textbf{Fecha de la API}: Actualización de los datos oficiales
  \item \textbf{Sección Gasolina 95}:
        \begin{itemize}
          \item Precio mínimo (verde)
          \item Precio máximo (rojo)
          \item Precio medio (negro)
          \item Top 5 estaciones más baratas
        \end{itemize}
  \item \textbf{Sección Diésel}: Misma estructura que Gasolina
  \item \textbf{Estadísticas generales}:
        \begin{itemize}
          \item Total de estaciones analizadas
          \item Estaciones sin precio de gasolina
          \item Estaciones sin precio de diésel
          \item Estaciones sin coordenadas GPS
        \end{itemize}
\end{itemize}

\subsection{Códigos de color}

\begin{table}[H]
  \centering
  \begin{tabular}{@{}ll@{}}
    \toprule
    \textbf{Color} & \textbf{Significado}          \\ \midrule
    Verde          & Precio por debajo de la media \\
    Rojo           & Precio por encima de la media \\
    Negro          & Precio igual a la media       \\ \bottomrule
  \end{tabular}
  \caption{Interpretación de colores en los precios}
\end{table}

\section{Mantenimiento}

\subsection{Ver logs}

\begin{lstlisting}[style=bash]
# Ver todo el log
cat log.txt

# Ver últimas 20 líneas
tail -20 log.txt

# Seguir el log en tiempo real
tail -f log.txt

# Buscar errores
grep -i "error\|fail" log.txt
\end{lstlisting}

\subsection{Limpiar archivos antiguos}

\begin{lstlisting}[style=bash]
# Borrar informes con mas de 30 dias
find informes/ -name "informe_*.html" -mtime +30 -delete
find informes/ -name "informe_*.txt" -mtime +30 -delete

# Borrar datos JSON con mas de 30 dias
find datos/ -name "*.json" -mtime +30 -delete
\end{lstlisting}

\subsection{Rotación de logs}

Para evitar que \code{log.txt} crezca indefinidamente:

\begin{lstlisting}[style=bash]
# Crear backup del log
cp log.txt log.txt.$(date +%Y%m%d)

# Vaciar el log actual
> log.txt

# O usar logrotate (avanzado)
\end{lstlisting}

\section{Solución de problemas}

\subsection{Error: Permission denied}

\textbf{Problema:} No se puede ejecutar el script

\textbf{Solución:}
\begin{lstlisting}[style=bash]
chmod +x analisis_json.sh
\end{lstlisting}

\subsection{Error: curl: command not found}

\textbf{Problema:} curl no está instalado

\textbf{Solución:}
\begin{lstlisting}[style=bash]
sudo apt-get install curl  # Ubuntu/Debian
\end{lstlisting}

\subsection{Error: jq: command not found}

\textbf{Problema:} jq no está instalado

\textbf{Solución:}
\begin{lstlisting}[style=bash]
sudo apt-get install jq  # Ubuntu/Debian
\end{lstlisting}

\subsection{Error HTTP 5xx}

\textbf{Problema:} La API del gobierno no responde

\textbf{Solución:} Esperar unos minutos y volver a intentarlo. El error quedará registrado en \code{log.txt}.

\section{Contacto y soporte}

Para reportar problemas o sugerir mejoras:

\begin{itemize}
  \item GitHub Issues: \url{https://github.com/juanfrantomas/shell-proyect/issues}
  \item Email: tu-email@ejemplo.com
\end{itemize}
